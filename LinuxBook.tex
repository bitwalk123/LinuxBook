\documentclass[a4j]{jreport}
\usepackage{alltt}
\usepackage{ascmac}
\usepackage{makeidx}
\usepackage{url}
\makeindex

\begin{document}

\title{Linuxの教科書}
\author{須栗歩人}
\maketitle

% 目次
\tableofcontents

% 部
\part{基本的なコマンド}

% 章
\chapter{ファイルとファイルシステム管理}

%%% ls
\section{\texttt{\index{ls}} --- ファイルやディレクトリの参照}

\texttt{ls}は、ファイルの一覧を表示するコマンドです。

\begin{itembox}[l]{使用例}
	\begin{alltt}\texttt{\$ ls
		ダウンロード  デスクトップ  ビデオ  画像
		テンプレート  ドキュメント  音楽    公開}\end{alltt}
\end{itembox}

%%% cp
\section{\texttt{\index{cp}} --- ファイルのコピー}

\texttt{cp}は、ファイルをある場所から別の場所へコピーするコマンドです。

\begin{itembox}[l]{使用例}
	\begin{alltt}\texttt{\$ cp \textit{filename1 filename2}}\end{alltt}
\end{itembox}

%%% mv
\section{\texttt{\index{mv}} --- ファイルの移動}

\texttt{mv}は、ファイルやディレクトリの移動、名前の変更をするコマンドです。

\begin{itembox}[l]{使用例}
	\begin{alltt}\texttt{\$ mv \textit{filename1 filename2}}\end{alltt}
\end{itembox}

%%% rm
\section{\texttt{\index{rm}} --- ファイルの削除}

\texttt{rm}は、ファイルシステムよりファイルを削除するコマンドです。

\begin{itembox}[l]{使用例}
	\begin{alltt}\texttt{\$ rm \textit{filename}}\end{alltt}
\end{itembox}

%%% pwd
\section{\texttt{\index{pwd}} --- 現在の作業ディレクトリ名の表示}

\texttt{pwd}は、現在の作業ディレクトリのフルパスを出力するコマンドです。

\begin{itembox}[l]{使用例}
	\begin{alltt}\texttt{\$ pwd
			/home/bitwalk}\end{alltt}
\end{itembox}

%%% cd
\section{\texttt{\index{cd}} --- ディレクトリの変更}

\texttt{cd}は、シェル等の現在の作業ディレクトリを変更するコマンドです。

\begin{itembox}[l]{使用例}
	\begin{alltt}\texttt{\$ cd ダウンロード}\end{alltt}
\end{itembox}

%%% mkdir
\section{\texttt{\index{mkdir}} --- ディレクトリの作成}

\texttt{mkdir}は、ディレクトリを作成するコマンドです。

\begin{itembox}[l]{使用例}
	\begin{alltt}\texttt{\$ mkdir \textit{dirname}}\end{alltt}
\end{itembox}

%%% cat
\section{\texttt{\index{cat}} --- ファイルの内容を表示}

\texttt{cat}は、ファイルを連結させたり表示したりするのに用いるコマンドです。

\begin{itembox}[l]{使用例}
	\begin{alltt}\texttt{\$ cat .bash_profile
			# .bash_profile
			
			# Get the aliases and functions
			if [ -f ~/.bashrc ]; then
			. ~/.bashrc
			fi
			
			# User specific environment and startup programs
			\$ 
		}\end{alltt}
\end{itembox}

%%% find
\section{\texttt{\index{find}} --- ファイルの検索}

\texttt{find}は、ファイルシステムの1つ以上のディレクトリツリー上で検索を行い、ユーザーが指定した基準にマッチするファイルを探すコマンドです。

\begin{itembox}[l]{使用例}
	\begin{alltt}\texttt{\$ }\end{alltt}
\end{itembox}

%%% which
\section{\texttt{\index{which}} --- コマンドのパスを表示}

\texttt{witch}は、指定したコマンドプログラムのフルパスを表示するコマンドです。

\begin{itembox}[l]{使用例}
	\begin{alltt}\texttt{\$ which cp
			/usr/bin/cp}\end{alltt}
\end{itembox}

%%% which
\section{\texttt{\index{man}} --- コマンドのパスを表示}

\texttt{man}は、指定したコマンドプログラムのマニュアルを表示するコマンドです。

\begin{itembox}[l]{使用例}
	\begin{alltt}\texttt{\$ man ls
	LS(1)                          ユーザーコマンド                          LS(1)
	
	名前
	ls - ディレクトリの内容をリスト表示する
	
	書式
	ls [オプション]... [ファイル]...
	
	説明
	FILE   (デフォルトは現在のディレクトリ)  に関する情報を一覧表示します。
	-cftuvSUX のいずれも指定されず、 --sort も指定されていない場合、 要素は
	アルファベット順でソートされます。
	
	長いオプションで必須となっている引数は短いオプションでも必須です。
	
	-a, --all
	. で始まる要素を無視しない
	
	-A, --almost-all
	. および .. を一覧表示しない
	
	--author                -l と合わせて使用した時、各ファイルの作成者を表
	示する
	Manual page ls(1) line 1 (press h for help or q to quit)
}\end{alltt}
\end{itembox}



%%% 関連図書
\begin{thebibliography}{99}
	\item Wikipedia(ウィキペディア) \url{https://ja.wikipedia.org/}
\end{thebibliography}

\printindex
\end{document}